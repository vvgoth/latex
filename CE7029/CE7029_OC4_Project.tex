\documentclass[a4paper, 11pt]{article}
\usepackage[letterpaper,margin=1in]{geometry}
\usepackage{blindtext}
\usepackage{lastpage}
\usepackage{fancyhdr}
\usepackage{xcolor}
\usepackage{setspace}
\usepackage{amsmath}
\usepackage{graphicx}
\usepackage{float}
\usepackage[small,bf,hypcap=true]{caption}
\newenvironment{Figure}
  {\par\medskip\noindent\minipage{\linewidth}%
   \captionsetup{type=figure}}
  {\endminipage\par\medskip}
\usepackage[hidelinks]{hyperref}
\usepackage{titlesec}
\usepackage{tocloft}

\renewcommand{\cftsecleader}{\cftdotfill{\cftdotsep}}

\graphicspath{{./Figures}}

% Configure the header
\pagestyle{fancy} % Enable fancy headers
\fancyhead[L]{CE 7029} % Left-aligned header
\fancyhead[C]{Numerical Modelling of Offshore Wind Turbines} % Centered header
\fancyhead[R]{-/-/2025} % Right-aligned header

\onehalfspacing

\begin{document}

\titleformat{\section}
  {\normalfont\bfseries\fontsize{12}{10}\selectfont}
  {\large\thesection.} 
  {0.3em}
  {}

\thispagestyle{empty}

\begin{figure}[H]
    \vspace{0.6cm}
    \centering
    \includegraphics[width=0.45\textwidth]{logo.png}
\end{figure}
\vspace{0.8cm}

\begin{center}
    \textbf{\LARGE Middle East Technical University}
    \vspace{0.3cm}

    \textbf{\LARGE Department of Civil Engineering}
    \vspace{0.5cm}

    \textbf{\Large 2024-2025 Spring Semester}
    \vspace{1.5cm}

    \textbf{\Large CE 7029 Numerical Modelling of Offshore Wind Turbines}
    \vspace{0.9cm}

    \textbf{\Large Semester Project Report}
    \vspace{1.5cm}

    \large Instructor:

    \large Assoc. Prof. Dr. Elif Oğuz
    \vspace{1.2cm}

    \large Submitted by:
    
    \large Bilge Kutay

    \large 2511798

\end{center}

\newpage
\renewcommand{\contentsname}{Table of Contents} % Rename "Contents" to "Table of Contents"
\begin{center}
    \tableofcontents
\end{center}
\newpage

\section{Introduction}

\hspace*{0.5cm}The demand for renewable energy has driven significant advancements in offshore wind energy technologies, with floating offshore wind turbines (FOWTs) emerging as a practical solution for deep-water energy generation. This report focuses on the numerical modeling and analysis of the OC4 semisubmersible platform. 

The analysis of the OC4 semisubmersible platform was conducted using two approaches. Initially, simulations were performed using a pre-existing OC4 model in OpenFAST, a simulation tool for FOWTs, to analyze the platform's dynamic behavior under various load cases. These load cases, based on the methodology outlined in the study by Robertson et al. (2014), titled "Offshore Code Comparison Collaboration Continuation within IEA Wind Task 30: Phase II Results Regarding a Floating Semisubmersible Wind System," included free decay tests and wave-induced responses. The results were processed using BEMRosetta to extract natural frequencies and generate response graphs, which were compared to the reference study to validate the findings.

In addition to using the pre-existing OC4 model, the platform was also recreated independently using Rhino. The hydrodynamic properties of this recreated model were analyzed using HAMS to generate WAMIT output files, which were then converted into OpenFAST input files. The same load cases were applied to the recreated model in OpenFAST, allowing for a direct comparison between the pre-existing and recreated models. This approach provided a deeper understanding of the platform's performance and validated the modeling process.

This report provides a detailed account of the methodologies and results obtained during the numerical modeling and analysis of these floating offshore wind systems. By applying consistent load cases to both pre-existing and recreated models, the findings contribute to a comprehensive understanding of the dynamic behavior and performance of FOWTs under various conditions.

\section{Article Review}
\hspace*{0.5cm}The study by Robertson et al. (2014) presents an analysis of the OC4 semisubmersible floating offshore wind system developed for the DeepCwind project (\autoref{fig:OC4}), comparing results from 21 different load cases (\autoref{tab:image_table}) simulated by 21 organizations using 19 distinct modeling tools. The research evaluates the platform’s dynamic behavior under progressively complex conditions, from basic system identification to combined environmental loads and damage scenarios. The study emphasizes the importance of accurate modeling in predicting the platform's response to environmental forces, including wind, waves, and currents. 

This report focuses on the cases of free decay tests and wave-induced responses, which are crucial for understanding the platform's dynamic behavior. The free decay tests involve subjecting the platform to initial displacements and allowing it to oscillate freely, while the wave-induced responses are analyzed by applying regular and irregular waves to the platform and measuring its response in terms of heave, pitch, and surge motions.
The results of these tests provide valuable insights into the platform's natural frequencies and response characteristics, which are essential for designing and optimizing floating offshore wind systems.
\vspace{0.5cm}

\begin{figure}[htbp]
    \begin{minipage}{0.47\textwidth}
        \centering
        \includegraphics[width=0.99\textwidth]{OC4.png}
        \caption{\small OC4-DeepCwind floating wind system design. (Robertson et al., 2014)}
        \label{fig:OC4}
    \end{minipage}
    \hfill
    \begin{minipage}{0.5\textwidth}
        \centering
        \includegraphics[width=0.97\textwidth]{characteristics.png}
        \caption{\small Summary of semisubmersible properties (Robertson et al., 2014)}
        \label{fig:characteristics}
    \end{minipage}
\end{figure}

\begin{table}[H]
    \centering
    \caption{Load cases run in OC4 Phase II. (Robertson et al., 2014)}
    \label{tab:image_table}
    \begin{tabular}{c}
        \includegraphics[width=1\textwidth]{table_3.png} \\
    \end{tabular}
\end{table}


\section{Pre-existing Model Analysis}

\hspace*{0.5cm}In this section, the analysis of the OC4 semisubmersible platform was conducted using a pre-existing model in OpenFAST called "5MW\_OC4Semi\_WSt\_WavesWN". The model was subjected to free decay and wave-induced load cases, namely Group 1.3X, 2.1, and 2.2, as outlined in the study by Robertson et al. (2014). The results were processed using BEMRosetta to extract natural frequencies and generate response graphs.

\subsection{Free Decay Tests}
\hspace*{0.5cm}The free decay tests involves subjecting the platform to initial displacements and allowing it to oscillate freely. Each of the four cases from 1.3a to 1.3d analyze the four individual degrees of freedom, specifically surge, heave, pitch, and yaw. The platform is given an initial displacement in the respective degree of freedom for each load case. The motion response of the platform was calculated using OpenFAST, and the natural frequencies were extracted from the time-domain data using the Fast Fourier Transform (FFT) method on BEMRosetta. The motion results of the 1.3a and 1.3b can be seen in Figures 3-14. The natural frequencies obtained from the tests for 6 degrees of freedom are shown in Figures 16-21 and consistent with the reference study, confirming the accuracy of the modeling approach \autoref{fig:nat_freq}.
\vspace{0.3cm}

%%1.3a figures
\begin{figure}[H]
    \begin{minipage}{0.48\textwidth}
        \centering
        \includegraphics[width=0.97\textwidth]{1.3a_surge.png}
        \caption{\small Surge free decay platform motion response for 1.3a (Robertson et al., 2014)}
        \label{fig:1.3a_surge}
    \end{minipage}
    \hfill
    \begin{minipage}{0.49\textwidth}
        \centering
        \includegraphics[width=1\textwidth]{1.3a_surge_mine.png}
        \caption{\small Surge free decay platform motion response for 1.3a}
        \label{fig:1.3a_surge_mine}
    \end{minipage}
\end{figure}

\begin{figure}[H]
    \begin{minipage}{0.48\textwidth}
        \centering
        \includegraphics[width=0.97\textwidth]{1.3a_heave.png}
        \caption{\small Heave free decay platform motion response for 1.3a (Robertson et al., 2014)}
        \label{fig:1.3a_heave}
    \end{minipage}
    \hfill
    \begin{minipage}{0.49\textwidth}
        \centering
        \includegraphics[width=1\textwidth]{1.3a_heave_mine.png}
        \caption{\small Heave free decay platform motion response for 1.3a}
        \label{fig:1.3a_heave_mine}
    \end{minipage}
\end{figure}

\begin{figure}[H]
    \begin{minipage}{0.48\textwidth}
        \centering
        \includegraphics[width=0.97\textwidth]{1.3a_pitch.png}
        \caption{\small Pitch free decay platform motion response for 1.3a (Robertson et al., 2014)}
        \label{fig:1.3a_pitch}
    \end{minipage}
    \hfill
    \begin{minipage}{0.49\textwidth}
        \centering
        \includegraphics[width=1\textwidth]{1.3a_pitch_mine.png}
        \caption{\small Pitch free decay platform motion response for 1.3a}
        \label{fig:1.3a_pitch_mine}
    \end{minipage}
\end{figure}

%%1.3b figures
\begin{figure}[H]
    \begin{minipage}{0.47\textwidth}
        \centering
        \includegraphics[width=0.97\textwidth]{1.3b_surge.png}
        \caption{\small Surge free decay platform motion response for 1.3b (Robertson et al., 2014)}
        \label{fig:1.3b_surge}
    \end{minipage}
    \hfill
    \begin{minipage}{0.5\textwidth}
        \centering
        \includegraphics[width=1\textwidth]{1.3b_surge_mine.png}
        \caption{\small Surge free decay platform motion response for 1.3b}
        \label{fig:1.3b_surge_mine}
    \end{minipage}
\end{figure}

\begin{figure}[H]
    \begin{minipage}{0.47\textwidth}
        \centering
        \includegraphics[width=0.97\textwidth]{1.3b_heave.png}
        \caption{\small Heave free decay platform motion response for 1.3b (Robertson et al., 2014)}
        \label{fig:1.3b_heave}
    \end{minipage}
    \hfill
    \begin{minipage}{0.5\textwidth}
        \centering
        \includegraphics[width=1\textwidth]{1.3b_heave_mine.png}
        \caption{\small Heave free decay platform motion response for 1.3b}
        \label{fig:1.3b_heave_mine}
    \end{minipage}
\end{figure}

\begin{figure}[H]
    \begin{minipage}{0.47\textwidth}
        \centering
        \includegraphics[width=0.97\textwidth]{1.3b_pitch.png}
        \caption{\small Pitch free decay platform motion response for 1.3b (Robertson et al., 2014)}
        \label{fig:1.3b_pitch}
    \end{minipage}
    \hfill
    \begin{minipage}{0.5\textwidth}
        \centering
        \vspace{-0.3cm}
        \includegraphics[width=1\textwidth]{1.3b_pitch_mine.png}
        \caption{\small Pitch free decay platform motion response for 1.3b}
        \label{fig:1.3b_pitch_mine}
    \end{minipage}
\end{figure}

When the graphs plotted for the surge, heave, and pitch motions are compared to the reference study, it is observed that the results are consistent in terms of the overall pattern of the motion response. Specifically, for the 1.3b case, the pitch motion response shows a very close resemblance to the distinct grouping using Morison’s equation for calculating the viscous drag (dotted or dash-dotted results) versus those using a quadratic drag matrix (solid line results).

%%natural frequencies

\begin{figure}[H]
    \centering
    \includegraphics[width=0.9\textwidth]{nat_freq.png}
    \caption{\small Full-system natural frequencies (Robertson et al., 2014)}
    \label{fig:nat_freq}
\end{figure}

\begin{figure}[H]
    \begin{minipage}{0.49\textwidth}
        \centering
        \includegraphics[width=1\textwidth]{nat_freq_surge.png}
        \caption{\small Surge natural frequency}
        \label{fig:nat_freq_surge}
    \end{minipage}
    \hfill
    \begin{minipage}{0.5\textwidth}
        \centering
        \includegraphics[width=1\textwidth]{nat_freq_sway.png}
        \caption{\small Sway natural frequency}
        \label{fig:nat_freq_sway}
    \end{minipage}
\end{figure}

\begin{figure}[H]
    \begin{minipage}{0.49\textwidth}
        \centering
        \includegraphics[width=1\textwidth]{nat_freq_heave.png}
        \caption{\small Heave natural frequency}
        \label{fig:nat_freq_heave}
    \end{minipage}
    \hfill
    \begin{minipage}{0.5\textwidth}
        \centering
        \includegraphics[width=1\textwidth]{nat_freq_roll.png}
        \caption{\small Roll natural frequency}
        \label{fig:nat_freq_roll}
    \end{minipage}
\end{figure}

\begin{figure}[H]
    \begin{minipage}{0.49\textwidth}
        \centering
        \includegraphics[width=1\textwidth]{nat_freq_pitch.png}
        \caption{\small Pitch natural frequency}
        \label{fig:nat_freq_pitch}
    \end{minipage}
    \hfill
    \begin{minipage}{0.5\textwidth}
        \centering
        \includegraphics[width=1\textwidth]{nat_freq_yaw.png}
        \caption{\small Yaw natural frequency}
        \label{fig:nat_freq_yaw}
    \end{minipage}
\end{figure}

The natural frequency analysis of the semisubmersible platform shows a distinction in the magnitude of these frequencies across the six degrees of freedom. Specifically, the natural frequencies for heave, roll and pitch were notably higher than those for surge, sway and yaw, with heave demonstrating highest frequency overall.

This difference originates from the restoring mechanisms associated with each degree of freedom. Heave, roll, and pitch are governed by strong hydrostatic restoring forces. For heave, this is primarily due to buoyancy, where vertical displacements lead to significant changes in the buoyant force. Roll and pitch experience restoring moments due to the platform's metacentric heights, which resist angular displacements. These restoring forces result in higher natural frequencies. In contrast, the restoring forces for surge and sway are minimal for small displacements, leading to very low natural frequencies. The yaw response shows restoring strength of similar but reduced magnitude to roll and pitch. This rotational stiffness is primarily from hydrodynamic sources like the geometry of the body, rather than the hydrostatic mechanisms that dominate roll and pitch (Jonkman, 2007).


\subsection{Wave-Induced Tests}
\hspace*{0.5cm}The wave-induced responses were analyzed by applying regular and irregular waves to the platform. The peak period of the waves were taken as 10 seconds, and the significant wave height was set to 6 meters. For the irregular waves, the JONSWAP spectrum was used to generate the wave field, with a peak wave shape parameter of 2.87. 


\subsubsection{Load Case 2.1}

\hspace{0.5cm}Load case 2.1 was used to analyze the platform’s response to regular waves, focusing on heave, pitch, surge motions, and the tension force at fairlead 2. The analysis was conducted using two approaches: one incorporating second-order drift effects through quadratic transfer functions (QTF) to account for low-frequency wave forces, and another excluding drift effects to isolate first-order wave responses. The results, processed using BEMRosetta, were compared to the reference study (Figures 22–29).

The platform's motion responses showed a strong resemblence to NREL's FAST v8 simulations when drift effects were neglected, as expected, since they did not utilize QTF-based second-order analysis in their calculations. This alignment confirms that both approaches produced equivalent first-order hydrodynamic responses. This consistency validates the first-order hydrodynamic modeling approach. However, when drift forces were introduced, surge motion showed significant sensitivity to QTF inclusion due to its weak hydrodynamic restoring forces. In contrast, heave and pitch motions remained virtually unaffected by second-order effects because their strong hydrostatic stiffness and higher natural frequencies (Figures 16, 18, and 20) make them more responsive to first-order wave excitations. The observed reduction in maximum fairlead 2 tension when neglecting QTF stems from this mooring point's direct coupling with surge motion along the y-axis; as the slow-drift components were removed, both surge displacement and consequent line tension decreased proportionally. These tension results aligned with dynamic mooring models, validating that the simulation properly accounted for the mooring system's mass, stiffness, and damping characteristics.

%% 2.1 figures
\begin{figure}[H]
    \begin{minipage}{0.48\textwidth}
        \centering
        \includegraphics[width=1\textwidth]{2.1_surge.png}
        \caption{\small Surge response for 2.1 (Robertson et al., 2014)}
        \label{fig:2.1_surge}
    \end{minipage}
    \hfill
    \begin{minipage}{0.51\textwidth}
        \centering
        \vspace{-0.3cm}
        \includegraphics[width=1\textwidth]{2.1_surge_mine.png}
        \caption{\small Surge response for 2.1}
        \label{fig:2.1_surge_mine}
    \end{minipage}
\end{figure}

\begin{figure}[H]
    \begin{minipage}{0.48\textwidth}
        \centering
        \includegraphics[width=1\textwidth]{2.1_heave.png}
        \caption{\small Heave response for 2.1 (Robertson et al., 2014)}
        \label{fig:2.1_heave}
    \end{minipage}
    \hfill
    \begin{minipage}{0.51\textwidth}
        \centering
        \vspace{-0.3cm}
        \includegraphics[width=1\textwidth]{2.1_heave_mine.png}
        \caption{\small Heave response for 2.1}
        \label{fig:2.1_heave_mine}
    \end{minipage}
\end{figure}

\begin{figure}[H]
    \begin{minipage}{0.48\textwidth}
        \centering
        \includegraphics[width=1\textwidth]{2.1_pitch.png}
        \caption{\small Pitch response for 2.1 (Robertson et al., 2014)}
        \label{fig:2.1_pitch}
    \end{minipage}
    \hfill
    \begin{minipage}{0.51\textwidth}
        \centering
        \vspace{-0.3cm}
        \includegraphics[width=1\textwidth]{2.1_pitch_mine.png}
        \caption{\small Pitch response for 2.1}
        \label{fig:2.1_pitch_mine}
    \end{minipage}
\end{figure}

\begin{figure}[H]
    \begin{minipage}{0.48\textwidth}
        \centering
        \includegraphics[width=1\textwidth]{2.1_fairten2.png}
        \caption{\small Fairlead 2 response for 2.1 (Robertson et al., 2014)}
        \label{fig:2.1_fairten2}
    \end{minipage}
    \hfill
    \begin{minipage}{0.51\textwidth}
        \centering
        \vspace{-0.3cm}
        \includegraphics[width=0.95\textwidth]{2.1_fairten2_mine.png}
        \caption{\small Fairlead 2 response for 2.1}
        \label{fig:2.1_fairten2_mine}
    \end{minipage}
\end{figure}

\subsubsection{Load Case 2.2}

\hspace{0.5cm}Load case 2.2 analyzes the platform's response to irregular waves, examining surge motion, pitch motion, and tower bending moment. The reference study shows significant variation in average responses between participants (Figure 30), caused by different modeling approaches, particularly whether drift effects were included. Results processed through BEMRosetta compare simulations with and without quadratic transfer function (QTF) analysis (Figures 31-33).

Similar to load case 2.1, the results show that the surge motion is highly sensitive to second-order drift effects. In contrast, pitch motion and tower bending moment show minimal changes. The consistent patterns observed in this analysis match well with participant results when accounting for each study's specific modeling choices. 

\begin{figure}[H]
    \centering
    \includegraphics[width=1\textwidth]{2.2.png}
    \caption{\small Average responses to irregular waves (Robertson et al., 2014)}
    \label{fig:2.2}
\end{figure}

\begin{figure}[H]
    \centering
    \includegraphics[width=0.78\textwidth]{2.2_surge.png}
    \caption{\small Surge response to irregular waves}
    \label{fig:2.2_surge}
\end{figure}

\begin{figure}[H]
    \centering
    \includegraphics[width=0.78\textwidth]{2.2_pitch.png}
    \caption{\small Pitch response to irregular waves}
    \label{fig:2.2_pitch}
\end{figure}

\begin{figure}[H]
    \centering
    \includegraphics[width=0.78\textwidth]{2.2_twr.png}
    \caption{\small Tower bending response to irregular waves (fore/aft direction)}
    \label{fig:2.2_twr}
\end{figure}

\section{Recreated Platform Model Analysis}

\hspace*{0.5cm}The OC4 semisubmersible platform was recreated independently using Rhino (Figure 34), and the WAMIT files of this model were created using the HAMS software. In Figures 35 and 36, the hydrostatic matrices of the models can be seen. The matrix for the recreated model includes some coupling forces caused by the meshing process. This results in slight differences between the outputs of the cases, which will be discussed in their respective sections. The same load cases were applied to the recreated model in OpenFAST, allowing for a direct comparison between the pre-existing (presented as the purple plot named "Ex") and recreated models (presented as the blue plot named "Re"). The comparison is done using the case without including the second-drift effects for the pre-existing model, as the HAMS software does not provide the required WAMIT outputs to use the QTF method.

\begin{figure}[H]
    \centering
    \includegraphics[width=1\textwidth]{rhino.png}
    \caption{\small Recreated OC4 semisubmersible platform in Rhino}
    \label{fig:rhino}
\end{figure}

\begin{figure}[H]
    \centering
    \includegraphics[width=0.95\textwidth]{hyd_st_org.png}
    \caption{\small Hydrostatic matrix of the pre-existing model}
    \label{fig:hyd_st_org}
\end{figure}

\begin{figure}[H]
    \centering
    \includegraphics[width=0.95\textwidth]{hyd_st_re.png}
    \caption{\small Hydrostatic matrix of the recreated model}
    \label{fig:hyd_st_re}
\end{figure}

\subsection{Free Decay Tests}
\hspace*{0.5cm}The free decay tests were conducted on the recreated model, and the results were compared to those obtained from the pre-existing model (Figures 35-40). The natural frequencies were extracted from the time-domain data using the Fast Fourier Transform (FFT) method on BEMRosetta. (Figures 41-46). 

\begin{figure}[H]
    \begin{minipage}{0.47\textwidth}
        \centering
        \includegraphics[width=1\textwidth]{1.3_surge_mine.png}
        \caption{\small Surge free decay platform motion response for 1.3a (recreated model vs. pre-existing model)}
        \label{fig:1.3a_surge_mine_recreated}
    \end{minipage}
    \hfill
    \begin{minipage}{0.49\textwidth}
        \centering
        \includegraphics[width=0.95\textwidth]{1.3a_heave_mine_1.png}
        \caption{\small Heave free decay platform motion response for 1.3a (recreated model vs. pre-existing model)}
        \label{fig:1.3a_heave_mine_recreated}
    \end{minipage}
\end{figure}

\begin{figure}[H]
    \begin{minipage}{0.47\textwidth}
        \centering
        \includegraphics[width=1\textwidth]{1.3a_pitch_mine_1.png}
        \caption{\small Pitch free decay platform motion response for 1.3a (recreated model vs. pre-existing model)}
        \label{fig:1.3a_pitch_mine_recreated}
    \end{minipage}
    \hfill
    \begin{minipage}{0.49\textwidth}
        \centering
        \includegraphics[width=0.95\textwidth]{1.3b_surge_mine_1.png}
        \caption{\small Surge free decay platform motion response for 1.3b (recreated model vs. pre-existing model)}
        \label{fig:1.3b_surge_mine_recreated}
    \end{minipage}
\end{figure}

\begin{figure}[H]
    \begin{minipage}{0.47\textwidth}
        \centering
        \includegraphics[width=1\textwidth]{1.3b_heave_mine_1.png}
        \caption{\small Heave free decay platform motion response for 1.3b (recreated model vs. pre-existing model)}
        \label{fig:1.3b_heave_mine_recreated}
    \end{minipage}
    \hfill
    \begin{minipage}{0.49\textwidth}
        \centering
        \includegraphics[width=0.95\textwidth]{1.3b_pitch_mine_1.png}
        \caption{\small Pitch free decay platform motion response for 1.3b (recreated model vs. pre-existing model)}
        \label{fig:1.3b_pitch_mine_recreated}
    \end{minipage}
\end{figure}

The graphs plotted for the surge motion in both load cases shows a very close resemblance to the pre-existing model, whereas the pitch and especially heave motion response shows differences in the amplitude of the oscillations. This is due to the differences in the modeling approaches and the volume calculated for the submerged part of the platform from the recreated model. The recreated model was created to closely match the dimensions and properties of the pre-existing model, but slight variations in the geometry may have led to these differences in response. The original model had a volume of 13,917 m³, while the recreated model had a volume of 13,603 m³. The difference in volume causes the bouyancy force to be different, which directly affects the heave motion response. The recreated model's heave motion response shows a higher amplitude than the pre-existing model, since the volume of the body is smaller. On the heave plot for the load case 1.3a, the green colored line represents the recreated model with the volume kept same as the original model (Figure 36). For this case, the recreated model's heave motion response is very close to the pre-existing model, confirming that the volume of the submerged part of the platform is a significant factor in determining the heave motion response. The recreated model's pitch motion response is also affected by the volume difference, but to a lesser extent than the heave motion response, which is presumably due to coupling effects.

%%nat frequencies recreated model
\begin{figure}[H]
    \begin{minipage}{0.47\textwidth}
        \centering
        \includegraphics[width=1\textwidth]{nat_freq_sway_1.png}
        \caption{\small Sway natural frequency (recreated model vs. pre-existing model)}
        \label{fig:nat_freq_sway_recreated}
    \end{minipage}
    \hfill
    \begin{minipage}{0.48\textwidth}
        \centering
        \includegraphics[width=1\textwidth]{nat_freq_surge_1.png}
        \caption{\small Surge natural frequency (recreated model vs. pre-existing model)}
        \label{fig:nat_freq_surge_recreated}
    \end{minipage}
\end{figure}

\begin{figure}[H]
    \begin{minipage}{0.47\textwidth}
        \centering
        \includegraphics[width=1\textwidth]{nat_freq_heave_1.png}
        \caption{\small Heave natural frequency (recreated model vs. pre-existing model)}
        \label{fig:nat_freq_heave_recreated}
    \end{minipage}
    \hfill
    \begin{minipage}{0.48\textwidth}
        \centering
        \includegraphics[width=1\textwidth]{nat_freq_roll_1.png}
        \caption{\small Roll natural frequency (recreated model vs. pre-existing model)}
        \label{fig:nat_freq_roll_recreated}
    \end{minipage}
\end{figure}

\begin{figure}[H]
    \begin{minipage}{0.47\textwidth}
        \centering
        \includegraphics[width=1\textwidth]{nat_freq_pitch_1.png}
        \caption{\small Pitch natural frequency (recreated model vs. pre-existing model)}
        \label{fig:nat_freq_pitch_recreated}
    \end{minipage}
    \hfill
    \begin{minipage}{0.48\textwidth}
        \centering
        \includegraphics[width=1\textwidth]{nat_freq_yaw_1.png}
        \caption{\small Yaw natural frequency (recreated model vs. pre-existing model)}
        \label{fig:nat_freq_yaw_recreated}
    \end{minipage}
\end{figure}


The natural frequencies extracted from the time-domain data using the Fast Fourier Transform (FFT) method on BEMRosetta were consistent with those obtained from the pre-existing model, confirming the accuracy of the recreated model. As discussed previously, the heave motion shows some difference in natural frequency, which is due to the difference in the volume of the submerged part of the platform (Figure 43). The recreated model's heave natural frequency is lower than the pre-existing model's, which is consistent with the heave motion response. Additionally, the yaw motion also presents a slight difference, which is likely a result of the different hydrostatic matrices caused by the modelling process (Figure 46). The other degrees of freedom show very close results to the pre-existing model.

\subsection{Wave-Induced Tests}

\hspace*{0.5cm}The wave-induced responses were analyzed by applying regular and irregular waves to the recreated model. The peak period of the waves were taken as 10 seconds, and the significant wave height was set to 6 meters. For the irregular waves, the JONSWAP spectrum was used to generate the wave field, with a peak wave shape parameter of 2.87. The results were compared to those obtained from the pre-existing model (No QTF case).

\subsubsection{Load Case 2.1}
\hspace*{0.5cm}The load case 2.1 was used to analyze the platform’s response to regular waves, focusing on heave, pitch, surge motions, and the tension force at fairlead 2. The analysis was conducted using the pre-existing model and the recreated model and compared (Figures 47-50).

\begin{figure}[H]
    \begin{minipage}{0.48\textwidth}
        \centering
        \includegraphics[width=1\textwidth]{2.1_surge_mine_1.png}
        \caption{\small Surge response for 2.1}
        \label{fig:2.1_surge_mine_recreated}
    \end{minipage}
    \hfill
    \begin{minipage}{0.5\textwidth}
        \centering
        \includegraphics[width=0.95\textwidth]{2.1_heave_mine_1.png}
        \caption{\small Heave response for 2.1}
        \label{fig:2.1_heave_mine_recreated}
    \end{minipage}
\end{figure}

\begin{figure}[H]
    \begin{minipage}{0.48\textwidth}
        \centering
        \includegraphics[width=1\textwidth]{2.1_pitch_mine_1.png}
        \caption{\small Pitch response for 2.1}
        \label{fig:2.1_pitch_mine_recreated}
    \end{minipage}
    \hfill
    \begin{minipage}{0.5\textwidth}
        \centering
        \includegraphics[width=0.95\textwidth]{2.1_fairten2_mine_1.png}
        \caption{\small Fairlead 2 response for 2.1}
        \label{fig:2.1_fairten2_mine_recreated}
    \end{minipage}
\end{figure}

The results show that the surge and pitch motions are very close to the pre-existing model, while the heave motion shows a difference in the amplitude of the oscillations. This is consistent with the free decay tests, where the heave motion response was affected by the volume and slight position difference between the two models, which causes the hydrostatic restoring matrix to have coupling forces between degrees of freedom. The fairlead 2 tension response also shows a difference in amplitude, which is due to the same reason as the heave response.

\subsubsection{Load Case 2.2}
\hspace*{0.5cm}The load case 2.2 analyzes the platform's response to irregular waves, examining surge motion, pitch motion, and tower bending moment. The results were compared to those obtained from the pre-existing model (Figures 53-55). 

\begin{figure}[H]
    \centering
    \includegraphics[width=0.7\textwidth]{2.2_surge_mine.png}
    \caption{\small Surge response to irregular waves (recreated model vs. pre-existing model)}
    \label{fig:2.2_surge_mine}
\end{figure}

\begin{figure}[H]
    \centering
    \includegraphics[width=0.7\textwidth]{2.2_pitch_mine.png}
    \caption{\small Pitch response to irregular waves (recreated model vs. pre-existing model)}
    \label{fig:2.2_pitch_mine}
\end{figure}

\begin{figure}[H]
    \centering
    \includegraphics[width=0.7\textwidth]{2.2_twr_mine.png}
    \caption{\small Tower bending response to irregular waves (fore/aft direction) (recreated model vs. pre-existing model)}
    \label{fig:2.2_twr_mine}
\end{figure}

Comparing the results of the recreated model to the pre-existing model, it can be observed that the mean values for both models align closely, with very similar patterns. The mean surge motion shows the largest difference with 18\% higher amplitude for the recreated model, while the pitch motion shows a 7\% lower amplitude and the tower bending moment shows a 1.4\% lower amplitude. These differences are due to the same reasons discussed previously, including the volume difference and the coupling forces in the hydrostatic matrix. 

\end{document}